\begin{document}
	
\begin{abstract}
  This is an example document for the \textsf{achemso} document
  class, intended for submissions to the American Chemical Society
  for publication. The class is based on the standard \LaTeXe\
  \textsf{report} file, and does not seek to reproduce the appearance
  of a published paper.

  This is an abstract for the \textsf{achemso} document class
  demonstration document.  An abstract is only allowed for certain
  manuscript types.  The selection of \texttt{journal} and
  \texttt{type} will determine if an abstract is valid.  If not, the
  class will issue an appropriate error.
\end{abstract}

% test test

\section{Introduction}
This is a paragraph of text to fill the introduction of the
demonstration file.  The demonstration file attempts to show the
modifications of the standard \LaTeX\ macros that are implemented by
the \textsf{achemso} class.  These are mainly concerned with content,
as opposed to appearance.

\section{Results and discussion}

\subsection{Outline}
% this is a comment
The document layout should follow the style of the journal concerned.
Where appropriate, sections and subsections should be added in the
normal way. If the class options are set correctly, warnings will be
given if these should not be present.


\subsection{Floats}

New float types are automatically set up by the class file.  The
means graphics are included as follows (\ref{sch:example}).  As
illustrated, the float is ``here'' if possible.
\begin{scheme}
  Your scheme graphic would go here: \texttt{.eps} format\\
  for \LaTeX\, or \texttt{.pdf} (or \texttt{.png}) for pdf\LaTeX\\
  \textsc{ChemDraw} files are best saved as \texttt{.eps} files;\\
  these can be scaled without loss of quality, and can be\\
  converted to \texttt{.pdf} files easily using \texttt{eps2pdf}.\\
  %\includegraphics{graphic}
  \caption{An example scheme}
  \label{sch:example}
\end{scheme}

\subsection{Math(s)}
test test test The \textsf{achemso} class does not load any particular additional
support for mathematics.  If the author \emph{needs} things like \textsf{amsmath}, they should be loaded in the preamble.  However, the basics should work fine.  Some inline material $y = mx + c$ blah 

\section{Experimental}

The usual experimental details should appear here.  This could
include a table, which can be referenced as \ref{tbl:example}. Notice
that the caption is positioned at the top of the table. Do not worry
about the appearance of the table: this will be altered during
production.
\begin{table}
  \caption{An example table}
  \label{tbl:example}
  \begin{tabular}{ll}
    \hline
    Header one & Header two \\
    \hline
    Entry one & Entry two \\
    Entry three & Entry four \\
    Entry five & Entry five \\
    Entry seven & Entry eight \\
    \hline
  \end{tabular}
\end{table}


The example file also loads the \textsf{mhchem} package, so
that formulas are easy to input: \texttt{\textbackslash
\ce\{H2SO4\}} gives \ce{H2SO4}.  See the use in the
bibliography file (when using titles in the references
section).

The use of new commands should be limited to simple things which will
not interfere with the production process.  For example,
\texttt{\textbackslash mycommand} has been defined in this example,
to give italic, monospaced text: \mycommand{some text}.

\acknowledgement

Thanks to Mats Dahlgren for version one of \textsf{achemso},
and Donald Arseneau for the code taken from \textsf{cite} to
move citations after punctuation.


\suppinfo

The entire \textsf{achemso} bundle is generated by running
\texttt{achemso.dtx} through \TeX. Running \LaTeX\ on the same file
will generate the general documentation for both the class and
package files.

\bibliography{achemso}
\end{document}


