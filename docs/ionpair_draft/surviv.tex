\documentclass[12pt]{article}
\usepackage{amsmath}
\usepackage{amssymb}
\usepackage{fullpage}

\begin{document}

\section{Survival probabilities}
The function that we want to calculate is as follows:

\[P_{surv}(t)=\frac{1}{N(t)}\sum_{\tau =1}^{T} N(\tau,\tau+t)\]

Where $T$ is the number of time steps, $N(t)$ is the number of
time intervals with length t, $N(\tau,\tau+t)$ is the number of
times state $s$ survived a length of time $(\tau, \tau+t)$

Note that $P(0)=1$. How is this related to the survival function?
(see Wolfram).


\end{document}
